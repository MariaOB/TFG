\chapter{Objetivos}

Los Diagramas de Voronoi están estrechamente relacionados con la Geometría Computacional por lo que su temática es apropiada para un trabajo final de este Doble Grado. Abarca tanto aspectos matemáticos de geometría ya que, de alguna forma, ellos almacenan toda la información referente a la proximidad entre puntos, como aspectos informáticos en cuanto a la implementación de algoritmos que se basan en su utilización.

Aunque es casi imposible presentar todos los resultados conocidos en los Diagramas de Voronoi y todo lo relacionado con ellos, nuestro objetivo es estudiar los fundamentos principales relacionados con los puntos de vista estructurales y algorítmicos. Además de ser una estructura de división de espacio versátil, los Diagramas de Voronoi también son estéticamente agradables y muchas personas se sienten atraídas por ellos, incluso con respecto a sus aspectos artísticos.

Para la consecución de este objetivo, es necesario profundizar en la comprensión de los conceptos teóricos asociados, algunos de los cuales están actualmente en desarrollo, y así adquirir los conocimientos necesarios que permitan desarrollar la librería.

A continuación mostramos cuáles son los objetivos generales de ambas partes, detallando los más específicos.

\vspace{0.5cm}
 
Desde el punto de vista matemático, hemos definido una serie de objetivos a alcanzar a lo largo del trabajo. Éstos se centran en el estudio de los Diagramas de Voronoi de diversas tipologías en espacios euclídeos, propiedades y algoritmos de obtención.
\begin{itemize}
	\item[{\ding{80}}] Comprender el concepto de Diagrama de Voronoi y algunas de sus propiedades más conocidas.
	\item[{\ding{80}}] Conocer la Triangulación de Delaunay junto con el algoritmo del vecino más cercano.
	\item[{\ding{80}}] Estudiar los algoritmos empleados para la elaboración de los Diagramas de Voronoi.  
\end{itemize}

Desde el punto de vista de la ingeniería informática marcamos como objetivo general la realización de una librería para el cálculo y gestión de los Diagramas de Voronoi en espacios euclídeos, incluyendo la entrada de puntos, la realización de la partición y otras funciones relevantes en diversas aplicaciones prácticas.
\begin{itemize}
	\item[{\ding{80}}] Especificar los requisitos funcionales y no funcionales.
	\item[{\ding{80}}] Elaborar una librería donde incluyamos lo necesario para implementar los Diagramas de Voronoi trabajando con el lenguaje de programación Java.
	\item[{\ding{80}}] Ser capaz de realizar una interfaz gráfica, con el uso de Java y estudiando las principales características de la programación de gráficos usando la tecnología Java2D, para hacer una aplicación en la que podamos ver, de manera visual, tanto lo explicado teóricamente como uno de los múltiples usos de estos diagramas.
\end{itemize}

Entender las propiedades de estos diagramas es la clave para su aplicación efectiva y para el desarrollo de la rápida construcción de algoritmos. Este problema es uno de los métodos de interpolación más simples que se basa en la distancia euclídea matemática. 
 
Para analizar y explicar con claridad este concepto, utilizaremos una base matemática relacionada directamente con la \underline{geometría} y, sobre todo, con la  \emph{distancia euclídea} y la \emph{Triangulación de Delaunay}. Por otro lado veremos varias formas de calcular una representación geométrica tanto de los Diagramas de Voronoi como de su dual (Triangulación de Delaunay) haciendo uso de algoritmos básicos de la informática.

 
Para entender estas ideas hemos hecho uso de \cite{clo}, ya que plasma toda la temática de una forma comprensible y bien estructurada. Además, nos ha facilitado la aproximación a nuevos conceptos que de otra forma hubiesen resultado más difíciles de interpretar.
 
Decidimos que la mejor forma de plasmar todo lo expuesto era realizar la implementación de una librería que nos permitiera visualizar lo referente al trabajo y entender de manera más dinámica todo lo relacionado con el cálculo y la gestión de estos diagramas. Para hacerlo, hemos usado el lenguaje de programación \emph{Java}.

