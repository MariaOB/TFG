\chapter{Conclusiones y Trabajos Futuros}


El objetivo general que ha marcado la realización de este Trabajo Fin de Grado ha sido doble: por un lado, el estudio de los Diagramas de Voronoi de diversas tipologías en espacios euclídeos, propiedades y algoritmos de obtención desde el punto de vista matemático; desde el punto de vista de la Ingeniería Informática, la realización de una librería para el cálculo y gestión de los Diagramas de Voronoi en espacios euclídeos de cualquier dimensión, incluyendo la entrada de puntos, la realización de la partición y otras funciones relevantes en diversas aplicaciones prácticas. 
\vspace{0.3cm}

A lo largo del documento hemos podido comprobar que se han cumplido estos objetivos. Los primeros objetivos en completarse fueron los matemáticos, ya que sin la base matemática no hubiese podido realizar la aplicación. Aunque, conforme iba estudiando todos los conceptos matemáticos, fui pensando en cómo implementarlo todo utilizando el lenguaje Java con el que ya había trabajado durante la carrera aunque, quizás, no a tan gran escala como lo hemos hecho en este proyecto. 
\vspace{0.3cm}

De forma más específica, estos objetivos se han alcanzado mediante la aportación de los siguientes resultados: 

\begin{itemize}
    \item En los Capítulos \ref{capi3} y \ref{capi4} se han introducido definiciones formales de los conceptos ``Diagramas de Voronoi'' y ``Triangulación de Delaunay", así como otras definiciones relevantes para entender estas ideas. Hemos estudiado algunas caracterizaciones, además de habernos concienciado del gran uso que se le puede dar a este tema y las numerosas aplicaciones que tienen en nuestro día a día.
    \item En el Capítulo \ref{algoritmos} se ha llevado a cabo una revisión de las diferentes técnicas y algoritmos relacionados con los Diagramas de Voronoi y su dual. Aquí hemos podido comprobar que hay algoritmos que, aunque tengan un orden de complejidad óptimo, la dificultad de implementación es muy elevada especialmente si los tenemos en cuenta en dimensiones elevadas.
    \item En el Capítulo \ref{capi6} hemos explicado la implementación de una librería específica para los Diagramas de Voronoi y hemos puesto en práctica los conocimientos adquiridos para mostrarlos de manera gráfica.
\end{itemize}


Desde mi punto de vista, el trabajo ha resultado interesante, ya que he conseguido unificar en un mismo proyecto matemáticas e informática y poner en práctica todo el estudio realizado sobre los Diagramas de Voronoi.
Con respecto a la parte matemática, el aprender un tema como los Diagramas de Voronoi y realizar un trabajo extenso de manera casi autosuficiente (ya que he contado siempre con la ayuda de mi tutor) me ha hecho ampliar mi conocimiento, capacidad de comprensión y estudio, además de enseñarme a afrontar un trabajo más amplio desde el principio y revisando la parte de documentación hasta el último detalle. El trabajar con temas con los que no he tratado a lo largo de la carrera y el contar con gran ambición para afrontar y aprender todo lo nuevo que se me presenta, me ha servido como motivación para el desarrollo de este trabajo.
En cuanto a la parte informática, el seguir aprendiendo sobre los lenguajes de Java y Python me ha parecido de gran utilidad ya que es un lenguaje que se está usando en la actualidad del mundo empresarial.
\vspace{0.3cm}

Además, como consecuencia de la investigación realizada se nos han planteado una serie de líneas de investigación que consideramos interesantes para afrontar en el futuro: 

\begin{itemize}
    \item Ampliación del estudio de los Diagramas de Voronoi.
    \item Estudiar los Diagramas de Voronoi generados a partir de los centros de los círculos (\cite{circulos}, \cite{gardenfor}).
    \item Extender la librería ya diseñada poder desarrollar distintas aplicaciones.
\end{itemize}