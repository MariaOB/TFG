\chapter{Introducción}

\pagestyle{fancy}


 El trabajo está dedicado a una estructura geométrica muy importante e influyente llamada \textbf{Diagrama de Voronoi}, cuyos orígenes, en la literatura occidental, nos llevan, al menos, al siglo XVII. Un Diagrama de Voronoi es una descomposición de un espacio métrico en regiones, de tal forma que, en dicha descomposición, a cada objeto se le asigna una región del espacio métrico formada por los puntos que están más cerca de él que de ninguno de los otros objetos. Es decir, divide el espacio en tantas regiones como puntos u objetos tengamos de tal forma que a cada punto se le asigna la región formada por todo lo que está más cerca de él que de ningún otro. En este proyecto abordaremos el estudio y evolución de los Diagramas de Voronoi, una de las estructuras fundamentales dentro de la Geometría Computacional.
 
 Varios nombres han sido dados a estos diagramas, dependiendo del dominio  (geografía, meteorología, biología, psicología). El matemático ruso Gueorgui  Voronoi, creador de estos diagramas, no fue el primero en estudiarlos de una manera más formal y sistemática, siendo el meteorólogo americano Alfred H. Thiessen el que desarrolló esta construcción geométrica que permitía construir particiones del plano euclídeo; los llamó \textbf{polígonos de Thiessen}. Estos polígonos también fueron estudiados por el matemático alemán Gustav Lejeune Dirichlet, que les dio el nombre de \textbf{teselación de Dirichlet}.
 
 Esta estructura es tan lógica e intuitiva que aparece cada vez que nos planteamos cuestiones relacionadas con problemas de proximidad y por eso muchos matemáticos se han dedicado a su estudio. Muchos autores señalan que esta estructura fue utilizada, incluso antes de Thiessen, en trabajos realizados por Descartes en 1644 sobre el cielo, cuando afirmaba que el Universo estaba formado por vórtices y plagado de éter y materia. René Descartes, en su trabajo sobre los principios de la filosofía, afirma que el sistema solar está constituido por vórtices y en sus ilustraciones muestra una descomposición del espacio en regiones convexas, cada una con una materia giratoria alrededor de una estrella fija. Aunque no especificó explícitamente la extensión de estas regiones, la idea parece ser la siguiente: tomemos un espacio y un conjunto de lugares o sitios dados en ese espacio junto con una noción de la influencia que un lugar ejerce sobre cada localización del espacio. Por tanto, la región de cada lugar abarca todos los puntos en los que la influencia es más fuerte. Fue John Snow, para muchos padre de la epidemiología moderna, el que los utilizó cuando se produjo el brote de cólera en Londres en 1854. John dedujo que la causa de la enfermedad era el consumo de aguas contaminadas por heces. Para ello, en un mapa, señaló la distribución de muertes por cólera y estudió dónde se encontraban las fuentes de agua potable de la ciudad y delimitó las “regiones de Voronoi” de cada una de esas bombas. Calculando la distancia entre la residencia de cada difunto y la bomba de agua más cercana, llegó a la conclusión de que la zona más afectada por la enfermedad se correspondía con la región de Voronoi asociada a la bomba de Broad Street, ya que en dicha región se dieron 73 de los 83 casos. Al retirar la fuente de agua de dicha zona, el brote de cólera se extinguió. 
 
 Tras la solución de este problema, el Diagrama de Voronoi ha tenido cada vez más uso y se ha ido desarrollando tanto en el plano como en el espacio (en dos y tres dimensiones). Es importante, tanto en la teoría como en la práctica, y es por eso que su éxito está presente en campos muy diversos como, por ejemplo, en Robótica, donde se puede usar para marcar los caminos que puede seguir un robot ajustando los puntos del diagrama a los obstáculos que podría encontrarse durante su trayectoria. Pero ésta no es la única aplicación en la que nos podemos encontrar presentes estos diagramas, ya que en casi cualquier ámbito en el que pensemos seguramente se estén usando de manera más o menos visible estas estructuras, teniendo cada una su propia noción de espacio, lugar e influencia: en el caso de la organización territorial a nivel de España para la división de comunidades autónomas; a nivel deportivo, estudiando las zonas más próximas a cada jugador para planear las jugadas; en la naturaleza podemos encontrarlos en las pieles de los animales y las hojas de los árboles o en zonas desérticas, donde las grietas del terreno se ajustan según los puntos de máxima humedad; en Arquitectura, cada vez están más demandadas a la hora de decorar espacios debido a su geometría, etc.
 
 Además de sus aplicaciones directas en todos los campos de la ciencia que hemos mencionado, los Diagramas de Voronoi pueden usarse para la resolución de numerosos problemas geométricos y teóricos. Por su estrecha relación con politopos (generalización a cualquier dimensión de un polígono bidimensional, o un poliedro tridimensional) e incluso con hiperplanos en dimensiones superiores, muchas preguntas (y respuestas) de conversión y geometría discreta se trasladan a estos diagramas. Además, la \textbf{Triangulación de Delaunay}, vista como un grafo combinatorio, está relacionada con varios grafos conectados.
 
 Por todo esto, parece razonable mantener la idea, durante todo lo que nos sigue, de que cada uno de nosotros quizás nos encontremos en una región que llamaremos nuestra, pero Voronoi estará presente en la de todos. Esta es la  principal intención de este trabajo, guiarnos al fascinante mundo de los Diagramas de Voronoi.
 
 